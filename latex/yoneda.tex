\documentclass[journal]{journal}

\usepackage[utf8]{inputenc}

\usepackage{amsmath, amsthm, amsfonts, amssymb}
\usepackage{mathrsfs}
\usepackage{bbm} 
\usepackage{bm} 

\usepackage[colorlinks=true,linkcolor=blue,citecolor=blue,urlcolor=blue]{hyperref}

\usepackage{url}

\usepackage{nameref}

\usepackage{makeidx}
\makeindex

\setlength{\parskip}{3pt}
\setlength{\parindent}{0pt}

\newcommand{\bcirc}{\bm{\circ}}

\newcommand{\defn}{\subsection{Definition}\begingroup\rm}
\newcommand{\ass}{\subsubsection{Assumptions}\begingroup\rm}
\newcommand{\laws}{\subsubsection{Laws}\begingroup\rm}
\renewcommand{\not}{\subsubsection{Notation}\begingroup\rm}
\newcommand{\subdefns}{\subsubsection{Definitions}\begingroup\rm}
\newcommand{\prop}{\subsubsection{Property}\begingroup\rm}
\newcommand{\prf}{\subsection{Proof}\begingroup\rm}
\newcommand{\lemm}{\subsection{Lemma}\begingroup\rm}
\newcommand{\corr}{\subsection{Corollary}\begingroup\rm}

\def\edefn{\endgroup\par\pagebreak[2]\addvspace{\medskipamount}}
\let\eass=\edefn
\let\elaws=\edefn
\let\enot=\edefn
\let\esubdefns=\edefn
\let\eprop=\edefn
\let\elemm=\edefn
\let\ecorr=\edefn
\let\eprf=\edefn

\newcounter{lister}

\newenvironment{labeledlist}[1]
{\begin{list}{{\rm#1\,--\,\arabic{lister}}}{\usecounter{lister}
\settowidth{\labelwidth}{#1--1}
\leftmargin\labelwidth \advance\leftmargin by \labelsep
}}
{\end{list}}

\def\blist#1{\begin{labeledlist}{#1}\setcounter{lister}{-1}}

\newcommand{\elist}{\end{labeledlist}}

\newcommand{\arrow}[2]{#1\!\!\rightarrow\!\!#2}
\newcommand{\arrowdown}[2]{#1 \rightarrow #2}

\newcommand{\function}[2]{\mbox{\bf #1} \!\!\Rightarrow\!\! \mbox{\bf #2}}

\newcommand{\nodefunction}[2]{#1 \!\!\Rightarrow\!\! #2}

\newcommand{\graphmorphism}[2]{#1 \rightarrow #2}
\newcommand{\nodemorphism}[2]{#1 \rightarrow #2}
\newcommand{\arrowmorphism}[2]{#1 \rightarrow #2}

\newcommand{\arrowcomposition}[2]{#1 \rightarrow #2}
\newcommand{\arrowunit}[2]{#1 \rightarrow #2}

\newcommand{\idarrow}[3]{#2 \!\rightarrow_{#1}\! #3}

\newcommand{\functor}[2]{#1 \rightarrow #2}

\newcommand{\transformation}[2]{#1 \rightarrow #2}

%\newcommand{\YF}[1]{\mbox{$Y$}^{\mbox{\scriptsize{$#1$}}}\!}
\newcommand{\YF}{\mbox{\bf YF}}
\newcommand{\yf}{\mbox{\bf yf}}

% \newcommand{\yf}[1]{\mbox{$y$}^{\mbox{\scriptsize{$#1$}}}}

\newcommand{\Arr}{\mbox{Arr}}

\newcommand{\FTA}{\mbox{{\bf F}2{\bf A}}}
\newcommand{\fta}{\mbox{{\normalsize {\bf f}}{\small 2}{\normalsize {\bf a}}}}
\newcommand{\cfta}[1]{\mbox{{\normalsize {\bf f}}{\small 2}{\normalsize {\bf a}}}_{#1}}

\newcommand{\YA}[1]{\mbox{Y\!}_{\mbox{\scriptsize{$#1$}}}\!}
\newcommand{\ya}[1]{\mbox{y\!}_{\mbox{\scriptsize{$#1$}}}}
\newcommand{\sya}[1]{\mbox{\scriptsize y}_{\mbox{\scriptsize{$#1$}}}}

\newcommand{\Z}{\mbox{\bf Z}}
\newcommand{\Y}{\mbox{\bf Y}}
\newcommand{\X}{\mbox{\bf X}}

\newcommand{\U}{\mbox{\bf U}}

\newcommand{\I}{\mbox{\bf I}}

\newcommand{\z}{\mbox{\bf z}}
\newcommand{\y}{\mbox{\bf y}}
\newcommand{\x}{\mbox{\bf x}}

\renewcommand{\u}{\mbox{\bf u}}

\renewcommand{\i}{\mbox{\bf i}}

\newcommand{\caf}{\mbox{\bf caf}}

\newcommand{\si}{\mbox{\bf {\scriptsize i}}}

\newcommand{\Set}{\mbox{\bf Set}}
\newcommand{\Type}{\mbox{\bf Type}}
\newcommand{\Fun}{\mbox{\bf Fun}}

\hyphenation{}

\pagestyle{empty}

\begin{document}

\title{
A pointfree Yoneda lemma\\
for\\
endofunctors\\
of\\
functional categories
}

\author{
Luc Duponcheel, \\ 
~\IEEEmembership{
mathematician and programmer, gardener and cyclist.
}
}

\thanks{
Luc Duponcheel is a retired mathematician and programmer. 
Nowadays, besides building a bridge between mathematics and programming,
he raises vegetables in his greenhouse and cycles with his wife all over Europe. 
}

\maketitle

\thispagestyle{empty}

\begin{abstract}

Category theory~\cite{book:MacLaneSaunders} is used in almost all areas of mathematics, and in some areas of
programming, more precisely pure effectful functional programming.

In mathematics, the Yoneda lemma suggests that instead of studying a category, one should study the category of all
functors from that category to $\Set$, the category whose objects are sets, and whose morphisms are functions. The lemma
uses only three concepts: category, functor and natural transformation. The lemma is pointful, it deals with elements of
sets.

In pure functional programming, one can write domain specific language libraries that model programming. Those libraries
model programs as morphisms of categories whose objects are types. Those categories, we call them functional categories,
have some specific properties. First, there exists a functor from $\Set$ to those categories that is the identity on
objects. Using that functor effectfree functions can be used as programs. Of course there may also exist effectful
programs. Second, programs themselves form a type.

This paper is a contribution to bridging the gap between mathematics and programming using the set-type, resp.
function-program correspondence. Category theory, almost inevitably, plays a fundamental role in doing
so~\cite{book:MacLaneSaunders}.

Functors from a functional category to $\Set$ can be composed with the functor above from $\Set$ to that functional
category to obtain an endofunctor of that functional category. It is natural to try to formulate and prove a Yoneda
lemma for such endofunctors. It is challenging to minimize the amount of concepts involved and to make both the
formulation and the proof of the lemma pointfree, only dealing with morphisms of that functional category.

This paper is the result of advancing insights, obained by writing a library in {\tt Scala}~\cite{book:Scala}. Any
programming language with a sufficiently powerful type system, for example {\tt Haskell}~\cite{book:Haskell}, could have
been used instead. Most notably its type system has to support higher kinded type classes~\cite{book:Pierce}.

\end{abstract}

\setcounter{section}{0}

\section{Introduction}\label{sec:introduction}

\subsection{Terminology and font conventions}\label{subsec:terminologynotationalconventions}

We use computing science category theory terminology. For example, 
\begin{itemize}
\item we use {\em node}\index{terminology!node} resp.\! {\em arrow} for {\em object} resp.\! {\em morphism},
\item we use {\em source}\index{terminology!source} resp.\! {\em target}\index{terminology!target} for {\em domain}
resp.\! {\em codomain}.     
\end{itemize}

We use the following font conventions
\begin{itemize}
\item for category specifications we use math font, like $C$, and $Z$, $Y$, \ldots\, , for nodes resp.
$\arrow{z}{y}$ \ldots\, , for arrows,
\item for category implementations, simply called categories, we use boldface font, like $\Set$, for the category of
sets, and $\Z$, $\Y$, \ldots\, , for sets resp. $\function{z}{y}$ \ldots\, , for functions.
\end{itemize}

\subsection{Notational assumptions and conventions}\label{subsec:notationalassumptionsandconvention}

We use notational assumptions for arrows of $C$.
\begin{itemize}
\item we assume that $\arrow{z}{y}$ \ldots\, , are names of arrows with source node $Z$ and target node $Y$,
\item we assume that $\arrow{z}{z}$ \ldots\, , are names of identity arrows with source node $Z$ and target node $Z$.           
\end{itemize} 

We use similar notational assumptions for functions of $\Set$.

Moreover,
\begin{itemize}       
\item we assume that {\bf z}, resp.\! {\bf y} \ldots\, , are names of elements of sets $\Z$, resp.\! $\Y$ \ldots\, .   
\end{itemize} 

We use notational conventions.  
\begin{itemize}
\item letters A and a, resp.\! F and f, resp.\! C and c, resp.\! Y and y refer to
"arrow", resp.\! "function", resp.\! "constant", resp.\! "Yoneda".
\end{itemize}

Function literals $\mbox{\bf z}\!\!\mapsto\!\!\mbox{\bf e}(\mbox{\bf z})$, where $\mbox{\bf e}(\mbox{\bf z})$ is an
expression containing $\mbox{\bf z}$ once, are, just like is often done when using programmatic notation, simply denoted 
$\mbox{\bf e}(\mbox{\bf \_})$.

\section{Preliminaries}\label{sec:preliminaries}

\setcounter{subsection}{0}

\defn\label{graph specification}
The {\em graph specification}\index{graph specification} $G$ declares
\blist{$G_{dec}$}
\item $G_0$, {\em nodes}\index{graph specification!node}, $Z$, $Y$, \ldots\, ,
\item $G_1$, {\em arrows}\index{graph specification!arrow}, $\arrow{z}{y}$, \ldots\, ,
\item $G_2$, {\em composable arrows}\index{graph specification!composabale arrows}
$\arrow{z}{y}$ and $\arrow{y}{x}$.
\elist
\edefn

Nodes do not need to form a set. $Arr_{G}(Z,Y)$, arrows with source node $Z$ and target node $Y$, form a set.

\not
$Arr_{G}(Z,Y)$ is also simply denoted $Arr(Z,Y)$.
\enot

\ass
{\em When using names like $\arrow{z}{y}$ we assume that $\arrow{z}{y} \in Arr(Z,Y)$}.
\eass

\defn\label{endoarrow}
An {\em endoarrow}\index{graph specification!endoarrow} is an arrow in $Arr(Z,Z)$.
\edefn

\defn\label{set graph}
The $\Set$ graph\index{set graph} defines
\blist{$\Set_{def}$}
\item $\Set_0$, {\em sets}\index{set graph!set}, $\Z$, $\Y$, \ldots\, ,
\item $\Set_1$, {\em functions}\index{set graph!function}, $\function{z}{y}$, \ldots\, . 
\elist
\edefn

Sets do not form a set. $\Fun(\Z,\Y)$, functions with source set $\Z$ and target set $\Y$ form a set.

\ass
{\em When using names like} $\function{z}{y}$ {\em we assume that} $\function{z}{y} \in \Fun(\Z,\Y)$.
\eass

\defn\label{graph morphism specification}
The {\em graph morphism specification}\index{graph morphism specification} $M: \graphmorphism{G}{H}$ declares
\blist{$M_{dec}$}
\item $M_0: \nodemorphism{G_0}{H_0}$,
\item $M_1: \arrowmorphism{G_1}{H_1}$.
\elist
\edefn

$M_0$ does not need to be a function. 

\not
Nodes $M_0(Z)$ are denoted $m(Z)$. Arrows $M_1(\arrow{z}{y})$ are denoted $m(\arrow{z}{y})$. 
\enot

\laws\label{graph morphism laws}
\blist{$M_{law}$}
\item if $\arrow{z}{y} \in Arr_{G}(Z,Y)$,\\
      then $m(\arrow{z}{y}) \in Arr_{H}(m(Z),m(Y))$.
\elist
\elaws

All $m: Arr_{G}(Z,Y) \rightarrow Arr_{H}(m(Z),m(Y))$ are functions.

\not
Arrows $m(\arrow{z}{y})$ are also denoted $\arrow{m(z)}{m(y)}$.
\enot

\defn\label{category specification}
The {\em category specification}\index{category specification} $C$ extends the graph specification. It delares
\blist{$C_{dec}$}
\item $c: \arrowcomposition{C_2}{C_1}$,
\item $u: \arrowunit{C_0}{C_1}$.
\elist
\edefn

$c$ is called {\em composition}\index{category specification!composition}.
$u$ is called {\em unit}\index{category specification!unit}.

\not
If $\arrow{z}{y}$ and $\arrow{y}{x}$ are composable arrows, then the arrow
$c(\arrow{z}{y},\arrow{y}{x})$ is denoted ${\arrow{y}{x}} \circ_{c} \arrow{z}{y}$, or, simply,
${\arrow{y}{x}} \circ \arrow{z}{y}$ and is called the
{\em composite arrow}\index{category specification!composite arrow} of $\arrow{z}{y}$ and $\arrow{y}{x}$, or, simply,
the {\em composite} of $\arrow{z}{y}$ and $\arrow{y}{x}$. If $Z$ is a node, then the arrow $u(Z)$ is denoted
$\idarrow{c}{z}{z}$, or, simply, $\arrow{z}{z}$ and is called the
{\em identity arrow}\index{category specification!identity arrow} of $Z$, or, simply, the {\em identity} of $Z$. 
\enot

\ass
{\em When using names like} $\idarrow{c}{z}{z}$ {\em we assume that} $\idarrow{c}{z}{z}$ {\em is the identity of} $Z$.
\eass

\laws\label{category laws}
\blist{$C_{law}$}
\item The source of ${\arrow{y}{x}} \circ_{c} \arrow{z}{y}$ is the source of $\arrow{z}{y}$,
\item The target of ${\arrow{y}{x}} \circ_{c} \arrow{z}{y}$ is the target of $\arrow{y}{x}$,
\item The source resp.\! target of $\arrow{y}{y}$ is $Y$,
\item The source resp.\! target of $\arrow{z}{z}$ is $Z$,
\item $(\arrow{x}{w} \circ_{c} \arrow{y}{x}) \circ_{c} \arrow{z}{y} = \arrow{x}{w} \circ_{c} (\arrow{y}{x}) \circ_{c} \arrow{z}{y})$
      ({\em associativity law}\index{category specification!associativity law}),
\item $\arrow{y}{y} \circ_{c} \arrow{z}{y} = \arrow{z}{y}$
      ({\em left identity law}\index{category specification!left identity law}),\label{category left identity law}      
\item $\arrow{z}{y} \circ_{c} \arrow{z}{z} = \arrow{z}{y}$
      ({\em right identity law}\index{category specification!right identity law}).\label{category right identity law}
\elist
\elaws

\defn\label{functor specification}
The {\em functor specification}\index{functor specification} $F: \functor{C}{D}$, where $C$ and $D$ are categories,
extends the graph morphism specification.
\edefn

\laws\label{functor laws}
\blist{$F_{law}$}
\item $f(\arrow{y}{x} \circ_{c} \arrow{z}{y}) = f(\arrow{y}{x}) \circ_{d} f(\arrow{z}{y})$ \\
      ({\em composition law}\index{functor specification!composition law}),
\item $f(\idarrow{c}{z}{z}) = \idarrow{d}{z}{z}$ 
      ({\em identity law}\index{functor specification!identity law}).
\elist
\elaws

\defn\label{endofunctor}
An {\em endofunctor}\index{functor specification!endofunctor} is a functor $F: \functor{C}{C}$.
\edefn

\defn\label{functor composition}
Given
\begin{itemize}
\item categories $C$, $D$ and $E$,
\item functors $F: \functor{C}{D}$ and $G: \functor{D}{E}$,  
\end{itemize}      
the {\em composed functor}\index{functor specification!composed functor} $G \bcirc F: \functor{C}{E}$ is defined as
follows:
\blist{$(G \bcirc F)_{def}$}
\item $(g \bcirc f)(Z) = g(f(Z))$,
\item $(g \bcirc f)(\arrow{z}{y}) = g(f(\arrow{z}{y}))$.
\elist
\edefn

\defn\label{identity functor}
The {\em identity functor}\index{functor specification!identity functor} $\I: \functor{C}{C}$ of category $C$ is defined
as follows
\blist{$\I_{def}$}
\item $\i(Z) = Z$,
\item $\i(\arrow{z}{y}) = \arrow{z}{y}$.
\elist
\edefn

\defn\label{Yoneda functor}
The {\em Yoneda functor}\index{Yoneda functor} $\YF\!_{Z}: \functor{C}{\Set}$ for node $Z$ of category $C$ is defined as
follows
\blist{$\YF\!_{Z}$}
\item $\yf_{z}(Y) = Arr(Z, Y)$,
\item $\yf_{z}(\arrow{y}{x}) = \arrow{y}{x} \circ \_$\, .
\elist
Note that $\yf_{z}(\arrow{y}{x})(\arrow{z}{y}) = \arrow{y}{x} \circ \arrow{z}{y}$.
\edefn

\defn\label{natural transformation specification}
The {\em natural transformation specification}\index{natural transformation specification}
$\alpha: \transformation{F}{G}$, where $F: \functor{C}{D}$ and $G: \functor{C}{D}$ are functors, declares
\blist{$\alpha_{dec}$}
\item arrows $\alpha_{z} \in Arr_{D}(f(Z),g(Z))$.
\elist
\edefn

\laws\label{natural transformation laws}
\blist{$\alpha_{law}$}
\item $\alpha_{y} \circ_{d} f(\arrow{z}{y}) = g(\arrow{z}{y}) \circ_{d} \alpha_{z}$
      ({\em commutativity law}\index{natural transformation specification!commutativity law}).\label{natural transformation commutativity law}
\elist
\elaws

\not
$\alpha_{z}$ is also, simply, denoted $\alpha$.
\enot

\defn\label{natural transformation composition}
Given
\begin{itemize}
\item a category $C$,
\item endofunctors, $F: \functor{C}{C}$, $G: \functor{C}{C}$, $H: \functor{C}{C}$,
\item natural transformations $\alpha: \transformation{F}{G}$ and $\beta: \transformation{H}{K}$,
\end{itemize} 
\edefn

the {\em composed natural transformation}\index{natural transformation specification!composed natural transformation}
$\beta \bcirc \alpha: \transformation{C}{C}$ is defined as follows:
\blist{$(\beta \bcirc \alpha)_{def}$}
\item $(\beta \bcirc \alpha)_{z} = \beta_{z} \circ \alpha_{z}$.
\elist

\defn\label{functor and natural transformation composition}
Given
\begin{itemize}
\item categories $C$, $D$ and $E$,
\item functors, $F: \functor{C}{D}$, $G: \functor{C}{D}$, $H: \functor{D}{E}$ and $K: \functor{D}{E}$,
\item natural transformations $\alpha: \transformation{F}{G}$ and $\beta: \transformation{H}{K}$,
\end{itemize} 

natural transformation $\beta\!F: \transformation{H \bcirc F}{K \bcirc F}$ is defined as follows: 
\blist{($\beta\!F)_{def}$}
\item $(\beta\!F)_{z} = \beta_{f(z)}$, and, 
\elist

natural transformation $H\!\alpha: \transformation{H \bcirc F}{H \bcirc G}$ is defined as follows:
\blist{($H\!\alpha)_{def}$}
\item $(H\!\alpha)_{z} = h(\alpha_{z})$.
\elist
\edefn

\defn\label{pre-triple specification} 
The {\em pre-triple specification}\index{pre-triple specification} $\pi\tau = (PT,\eta)$ for category $C$ declares
\blist{$\pi\tau_{dec}$}
\item an endofunctor $PT: \functor{C}{C}$,
\item a natural transformation $\eta: \transformation{\I}{PT}$.
\elist
$\eta$ is called {\em unit}\index{pre-triple specification!unit}.
\edefn

\defn\label{triple specification} 
The {\em triple specification}\index{triple specification} $\tau = (T,\mu,\eta)$ for category $C$ declares
\blist{$\tau_{dec}$}
\item a pre-triple $(T,\eta)$
\item a natural transformation $\mu: \transformation{T \bcirc T}{T}$.
\elist
$\mu$ is called {\em multiplication}\index{triple specification!multiplication}.
\edefn

Note that $\I = T^{0}$, $T = T^{1}$ and $T \bcirc T = T^{2}$.

\laws\label{triple laws}
\blist{$\tau_{law}$}
\item $\mu_{z} \circ_{c} (\eta T)_{z} = \arrow{t(z)}{t(z)}$
({\em left identity law}\index{triple specification!left identity law}),
\item $\mu_{z} \circ_{c} (T \eta)_{z} = \arrow{t(z)}{t(z)}$
({\em right identity law}\index{triple specification!right identity law}),
\item $\mu_{z} \circ_{c} (T \mu)_{z} = \mu_{z} \circ_{c} (\mu T)_{z}$
({\em multiplication law}\index{triple specification!multiplication law}).
\elist
\elaws

\section{Content}\label{sec:content}

\setcounter{subsection}{0}

\defn\label{pre-functional category specification}
The {\em pre-functional category specification}\index{pre-functional category specification} $P\!F\!C$, whose nodes are
the nodes of $\Set$, extends the category specification. 

\not
Although nodes of $P\!F\!C$ are sets, we will, by abuse of notation, denote them as nodes, $Z$, $Y$, \ldots\, \,\,
(and the same for their elements, $z$, $y$, \ldots\, ).
\enot

It declares
\blist{$P\!F\!C_{dec}$}
\item a functor $\FTA$ from $\Set$ to $P\!F\!C$ that is the identity on nodes.
\elist

We will use the following definitions
\blist{$P\!F\!C_{def}$}
\item $\Y\!_{Z}$, the {\em Yoneda endofunctor for $Z$}\index{pre-functional category specification!Yoneda endofunctor}, is 
      $\FTA \bcirc \YF\!_{Z}$,
\item $\caf \in \Fun(Y,Arr(Z, Y))$ is $y \mapsto \fta(z \mapsto y)$\index{pre-functional category specification!constant arrow function},
\item $\eta_{z} \in Arr(\I(Z),\Y\!_{U}(Z))$ is $\fta(\caf)$, where $U$ is a terminal node (singleton set) of the category $\Set$.
\elist

Note that $\fta$ stands for function to arrow, and $\caf$ stands for constant arrow function.

From a functional programming viewpoint, it is instructive to think about arrows $\fta(\nodefunction{z}{y})$ as {\em pure
functions}\index{pre-functional category specification!pure function} (a.k.a {\em effectfree programs}) and other
arrows as {\em effectful programs}\index{pre-functional category specification!effectful program} (a.k.a. 
{\em impure functions}).
\edefn

\not
$\fta$, mapping functions of $\Fun(Z, Y)$ to arrows of $Arr(Z, Y)$ for node $Z$, is, when, in our opinion, instructive,
explicitly denoted $\cfta{y}$.
\enot

\laws\label{pre-functional category laws}
\blist{$P\!F\!C_{law}$}
\item $\eta: \I \rightarrow \Y\!_{U}$ is a {\em natural} transformation, in other words, $(\Y\!_{U},\eta)$ is a 
      pre-triple\index{pre-functional category specification!$(\Y\!_{U},\eta)$ pre-triple}.
\item $\caf(\arrow{u}{z}) = \eta \circ \arrow{u}{z}$
      ({\em $\eta$ law}\index{pre-functional category specification!$\eta$ law})
\elist
\elaws

\prop
The following property can easily be proved
\blist{app}
\item $\fta(\nodefunction{z}{y}) \circ \caf(z) = \caf(\nodefunction{z}{y}(z))$ \\
      ({\em pointfree application}\index{pre-functional category specification!pointfree application}).\label{pointfree application property}
\elist
\eprop

\prop
The following property can easily be proved
\blist{$\Y$}
\item $\y\!_z(\arrow{y}{x}) \circ \caf(\arrow{z}{y}) = \caf(\arrow{y}{x} \circ \arrow{z}{y})$ \\
      ({\em pointfree Yoneda}\index{pre-functional category specification!pointfree Yoneda}).\label{pointfree Yoneda property}
\elist
\eprop

\subsection{Pointfree Yoneda lemma for pre-functional categories}

We are ready now for our pointfree Yoneda lemma for endofunctors of pre-functional categories.

\lemm\label{Pointfree Yoneda lemma for pre-functional categories}

For all pre-functional categories $P\!F\!C$, nodes $Z$ of $P\!F\!C$, and endofunctors $G$ of $P\!F\!C$, natural
transformations $\eta G \bcirc \beta: \Y\!_{Z} \rightarrow \Y\!_{U} \bcirc G$, for natural tranformations
$\beta: \Y\!_{Z} \rightarrow G$, correspond with arrows of $(\y\!_{u} \!\bcirc g)(Z)$.

On the one hand, if $\beta: \Y\!_{Z} \rightarrow G$ is a natural transformation, and
\begin{itemize} 
\item $\arrow{u}{g(z)} \in (\y\!_{u} \!\bcirc g)(Z)$ is defined as $\beta \circ \caf(\arrow{z}{z})$,
\end{itemize} 
then, $\gamma: \Y\!_{Z} \rightarrow \Y\!_{U} \!\bcirc G$, defined as
$\gamma = \fta(g(\_) \circ \arrow{u}{g(z)})$, is equal to
$\eta G \bcirc \beta$. 

On the other hand, if $\arrow{u}{g(z)} \in (\y\!_{u} \!\bcirc g)(Z)$, and
\begin{itemize}
\item $\gamma: \Y\!_{Z} \rightarrow \Y\!_{U} \!\bcirc G$ is defined as
$\gamma = \fta(g(\_) \circ \arrow{u}{g(z)})$,
\end{itemize}
then $\gamma$, is a natural transformation.
\elemm

Both statements hold modulo composing with an arrow $\caf(\arrow{z}{y})$.

\defn\label{functional category specification}
The {\em functional category specification}\index{functional category specification} $F\!C$ extends the pre-functional
category specification. It declares
\blist{$F\!C_{dec}$}
\item natural transformation $\mu: (\Y\!_{U} \!\bcirc \Y\!_{U}) \rightarrow \Y\!_{U}$.
\elist
\edefn

\laws\label{functional category laws}
\blist{$F\!C_{law}$}
\item $(\Y\!_{U}\!,\mu, \eta)$ is a triple\index{functional category specification!$(\Y\!_{U},\mu, \eta)$ triple}.
\elist
\elaws

\prop\label{functional category mu property}
The following property can easily be proved
\blist{$\mu_{prop}$}
\item $\mu \circ \caf(\arrow{u}{(\arrow{u}{z})}) = \arrow{u}{(\arrow{u}{z})}$
      ({\em $\mu$ property}\index{functional category specification!$\mu$ property})
\elist 
\eprop

\subsection{Pointfree Yoneda lemma for functional categories}

We are ready now for our pointfree Yoneda lemma for endofunctors of functional categories.

\lemm\label{Pointfree Yoneda lemma for functional categories}
For all functional categories $F\!C$, nodes $Z$ of $F\!C$, and endofunctors $G$ of $F\!C$, natural transformations
$\beta: \Y\!_{Z} \rightarrow \Y\!_{U} \bcirc G$ correspond with arrows of $(\y\!_{u} \!\bcirc \y\!_{u} \!\bcirc g)(Z)$. 

On the one hand, if $\beta: \Y\!_{Z} \rightarrow \Y\!_{U} \bcirc G$ is a natural transformation, and
\begin{itemize} 
\item $\arrow{u}{(\arrow{u}{g(z)})} \in (\y\!_{u} \!\bcirc \y\!_{u} \!\bcirc g)(Z)$ is defined as 
      $\beta \circ \caf(\arrow{z}{z})$,
\end{itemize} 
then, $\gamma: \Y\!_{Z} \,\rightarrow \Y\!_{U} \bcirc G$, defined as \\ 
$\gamma = \mu G \circ \fta((\y\!_{u} \!\bcirc g)(\_) \circ \arrow{u}{(\arrow{u}{g(z)})})$,
is equal to $\beta$. 

On the other hand, if $\arrow{u}{(\arrow{u}{g(z)})} \in (\y\!_{u} \bcirc \y\!_{u} \!\bcirc g)(Z)$, and
\begin{itemize}
\item $\gamma: \Y\!_{Z} \,\rightarrow \Y\!_{U} \bcirc G$ is defined as \\
$\gamma = \mu G \circ \fta((\y\!_{u} \!\bcirc g)(\_) \circ \arrow{u}{(\arrow{u}{g(z)})})$,
\end{itemize}
then $\gamma$, is a natural transformation.
\elemm

Both statements hold modulo composing with an arrow $\caf(\arrow{z}{y})$.

\appendices

\section{Proof of first lemma}

\prf
\setcounter{equation}{0}
On the one hand, if $\arrow{z}{y} \in Arr(Z,Y)$, then
\begin{eqnarray*}
&   & \!\!\!\!\!\!\!\!\!\!\!\!\!\!\!\! \eta G \circ \beta \circ \caf(\arrow{z}{y}) \\
&   & \mbox{right identity law for } Arr \\      
& = & \eta G \circ \beta \circ \caf(\arrow{z}{y} \circ \arrow{z}{z}) \\
&   & \mbox{pointfree Yoneda law for } (\Y\!_U, \eta) \\      
& = & \eta G \circ \beta \circ \y\!_{z}(\arrow{z}{y}) \circ \caf(\arrow{z}{z}) \\
&   & \mbox{natural transformation law for } \eta G \circ \beta \\      
& = & (\y\!_{u} \!\bcirc g)(\arrow{z}{y}) \circ \eta G \circ \beta \circ \caf(\arrow{z}{z}) \\
&   & \mbox{definition  } \arrow{u}{g(z)} \\      
& = & (\y\!_{u} \!\bcirc g)(\arrow{z}{y}) \circ \eta G \circ \arrow{u}{g(z)} \\
&   & \eta \mbox{ law for } (\Y\!_U, \eta) \\      
& = & (\y\!_{u} \!\bcirc g)(\arrow{z}{y}) \circ \caf(\arrow{u}{g(z)}) \\
&   & \mbox{function composition for } \y\!_{u} \mbox{ and } g \\      
& = & \y\!_{u}(g(\arrow{z}{y})) \circ \caf(\arrow{u}{g(z)}) \\
&   & \mbox{pointfree Yoneda law for } (\Y\!_U, \eta) \\      
& = & \caf(g(\arrow{z}{y}) \circ \arrow{u}{g(z)}) \\
&   & \mbox{pointfree application law for } (\Y\!_U, \eta) \\      
& = & \cfta{y}(g(\_) \circ \arrow{u}{g(z)}) \circ \caf(\arrow{z}{y}) \\
&   & \mbox{definition } \gamma \\      
& = & \gamma \circ \caf(\arrow{z}{y}) 
\end{eqnarray*}
\vspace{6pt}
\setcounter{equation}{0}

On the other hand, if $\arrow{z}{y} \in Arr(Z,Y)$, then
\begin{eqnarray*}
&   & \!\!\!\!\!\!\!\!\!\!\!\!\!\!\!\! (\y\!_{u} \!\bcirc g)(\arrow{y}{x}) \circ \gamma \circ \caf(\arrow{z}{y}) \\
&   & \mbox{definition } \gamma \\
& = & (\y\!_{u} \!\bcirc g)(\arrow{y}{x}) \circ \cfta{y}(g(\_) \circ \arrow{u}{g(z)}) \circ \caf(\arrow{z}{y}) \\
&   & \mbox{pointfree application law for } (\Y\!_U, \eta) \\      
& = & (\y\!_{u} \!\bcirc g)(\arrow{y}{x}) \circ \caf(g(\arrow{z}{y}) \circ \arrow{u}{g(z)}) \\
&   & \mbox{function composition for } \y\!_{u} \mbox{ and } g \\      
& = & \y\!_{u}(g(\arrow{y}{x})) \circ \caf(g(\arrow{z}{y}) \circ \arrow{u}{g(z)}) \\
&   & \mbox{pointfree Yoneda law for } (\Y\!_U, \eta) \\      
& = & \caf(g(\arrow{y}{x}) \circ g(\arrow{z}{y}) \circ \arrow{u}{g(z)}) \\
&   & \mbox{composition law for } g \\
& = & \caf(g(\arrow{y}{x} \circ \arrow{z}{y}) \circ \arrow{u}{g(z)}) \\
&   & \mbox{pointfree application law for } (\Y\!_U, \eta) \\      
& = & \cfta{x}(g(\_) \circ \arrow{u}{g(z)}) \circ \caf(\arrow{y}{x} \circ \arrow{z}{y}) \\
&   & \mbox{pointfree Yoneda law for } (\Y\!_U, \eta) \\      
& = & \cfta{x}(g(\_) \circ \arrow{u}{g(z)}) \circ \y\!_{z}(\arrow{y}{x}) \circ ca(\arrow{z}{y}) \\
&   & \mbox{definition } \gamma \\
& = & \gamma \circ \y\!_{z}(\arrow{y}{x}) \circ \caf(\arrow{z}{y})
\end{eqnarray*}
\eprf

\newpage

\section{Proof of second lemma}

\prf
On the one hand, if $\arrow{z}{y} \in Arr(Z,Y)$, then
\setcounter{equation}{0}
\begin{eqnarray*}
&   & \!\!\!\!\!\!\!\!\!\!\!\!\!\!\!\! \beta \circ \caf(\arrow{z}{y}) \\      
&   & \mbox{right identity law for } Arr \\      
& = & \beta \circ \caf(\arrow{z}{y} \circ \arrow{z}{z}) \\   
&   & \mbox{pointfree Yoneda law for } (\Y\!_U, \eta) \\      
& = & \beta \circ \y\!_{z}(\arrow{z}{y}) \circ \caf(\arrow{z}{z}) \\      
&   & \mbox{natural transformation law for } \beta \\      
& = & (\y\!_{u} \!\bcirc g)(\arrow{z}{y}) \circ \beta \circ \caf(\arrow{z}{z}) \\      
&   & \mbox{definition } \arrow{u}{(\arrow{u}{g(z)})} \\
& = & (\y\!_{u} \!\bcirc g)(\arrow{z}{y}) \circ \arrow{u}{(\arrow{u}{g(z)})} \\      
&   & \mu \mbox{ property for } (\Y\!_U, \mu, \eta) \\      
& = & \mu G \circ \caf((\y\!_{u} \!\bcirc g)(\arrow{z}{y}) \circ \arrow{u}{(\arrow{u}{g(z)})}) \\      
&   & \mbox{pointfree application law for } (\Y\!_U, \eta) \\      
& = & \mu G \circ \cfta{y}((\y\!_{u} \!\bcirc g)(\_) \circ \arrow{u}{(\arrow{u}{g(z)})}) \circ ca(\arrow{z}{y}) \\      
&   & \mbox{definition } \gamma \\
& = & \gamma \circ \caf(\arrow{z}{y}) 
\end{eqnarray*}
\vspace{6pt}
\setcounter{equation}{0}

On the other hand, if $\arrow{z}{y} \in Arr(Z,Y)$, then
\begin{eqnarray*}
&   & \!\!\!\!\!\!\!\!\!\!\!\!\!\!\!\! (\y\!_{u} \!\bcirc g)(\arrow{y}{x}) \circ \gamma \circ \caf(\arrow{z}{y}) \\
&   & \mbox{definition } \gamma \\
& = & (\y\!_{u} \!\bcirc g)(\arrow{y}{x}) \circ \mu G \circ \cfta{y}((\y\!_{u} \!\bcirc g)(\_) \circ \arrow{u}{(\arrow{u}{g(z)})}) \circ \caf(\arrow{z}{y}) \\
&   & \mbox{pointfree application law for } (\Y\!_U, \eta) \\      
& = & (\y\!_{u} \!\bcirc g)(\arrow{y}{x}) \circ \mu G \circ \caf((\y\!_{u} \!\bcirc g)(\arrow{z}{y}) \circ \arrow{u}{(\arrow{u}{g(z)})}) \\
&   & \mu \mbox{ property for } (\Y\!_U, \mu, \eta) \\      
& = & (\y\!_{u} \!\bcirc g)(\arrow{y}{x}) \circ (\y\!_{u} \!\bcirc g)(\arrow{z}{y}) \circ \arrow{u}{(\arrow{u}{g(z)})} \\
&   & \mbox{composition law for } \y\!_{u} \!\bcirc g \\
& = & (\y\!_{u} \!\bcirc g)(\arrow{y}{x} \circ \arrow{z}{y}) \circ \arrow{u}{(\arrow{u}{g(z)})} \\  
&   & \mu \mbox{ property for } (\Y\!_U, \mu, \eta) \\      
& = & \mu G \circ \caf((\y\!_{u} \!\bcirc g)(\arrow{y}{x} \circ \arrow{z}{y}) \circ \arrow{u}{(\arrow{u}{g(z)})}) \\  
&   & \mbox{pointfree application law for } (\Y\!_U, \eta) \\      
& = & \mu G \circ \cfta{x}((\y\!_{u} \!\bcirc g)(\_) \circ \arrow{u}{(\arrow{u}{g(z)})}) \circ \caf(\arrow{y}{x} \circ \arrow{z}{y}) \\  
&   & \mbox{definition } \gamma \\
& = & \gamma \circ \caf(\arrow{y}{x} \circ \arrow{z}{y}) \\  
&   & \mbox{pointfree Yoneda law for } (\Y\!_U, \eta) \\      
& = & \gamma \circ \y\!_{z}(\arrow{y}{x}) \circ \caf(\arrow{z}{y}) 
\end{eqnarray*}
\eprf

\begin{thebibliography}{1}

\bibitem{book:BarrWells}
M.~Barr and Ch. Wells, \emph{Category theory for computing science}, 2nd~ed.\hskip 1em plus
0.5em minus 0.4em\relax Prentice-Hall International Series in Computer Science, 1999.

\bibitem{book:MacLaneSaunders}
Mac Lane, Saunders, \emph{Categories for the Working Mathematician}, vol-5, 2nd~ed.\hskip 1em plus
0.5em minus 0.4em\relax Springer-Verlag, 1998.

\bibitem{book:Pierce}
Benjamin C. Pierce, \emph{Types and Programming Languages}.\hskip 1em plus
0.5em minus 0.4em\relax MIT Press, 2002.

\bibitem{book:Haskell}
Simon Peyton Jones, \emph{Haskell 98 language and libraries: the Revised Report}.\hskip 1em plus
0.5em minus 0.4em\relax Cambridge University Press, 2003.

\bibitem{book:Scala}
Martin Odersky, Lex Spoon, Bill Venners, and Frank Sommers, \emph{Programming in Scala}, 5th~ed.\hskip 1em plus
0.5em minus 0.4em\relax Artima, 2021.
  
\end{thebibliography}

\vspace{-600pt}

\begin{IEEEbiographynophoto}{Luc Duponcheel}
has worked as a mathematics researcher at various universities of Belgium and The Netherlands.
Later on he worked as an international Java programming instructor for Sun Microsystems.
At the end of his career he worked as an independent Scala software consultant.
All his life his end-of-day hobby was to build a bridge between mathematics and programming.
\end{IEEEbiographynophoto}

\end{document}

