\documentclass[11pt]{beamer}

\usepackage[utf8]{inputenc}

\usetheme{Singapore}

\definecolor{DarkRed}{rgb}{0.50,0,0}
\definecolor{DarkGreen}{rgb}{0,0.50,0}
\definecolor{DarkBlue}{rgb}{0,0,0.50}
\definecolor{Black}{rgb}{0,0,0}

\newcommand{\ttdr}[1]{{\tt{\color{DarkRed} #1}}}
\newcommand{\emdr}[1]{{\em{\color{DarkRed} #1}}}
\newcommand{\ttdg}[1]{{\tt{\color{DarkGreen} #1}}}
\newcommand{\emdg}[1]{{\em{\color{DarkGreen} #1}}}
\newcommand{\ttdb}[1]{{\tt{\color{DarkBlue} #1}}}
\newcommand{\emdb}[1]{{\em{\color{DarkBlue} #1}}}

\setbeamertemplate{frametitle}{
\begin{centering}
{\Large \textbf{\textmd{\insertframetitle}}}
\end{centering}
}

\setbeamertemplate{navigation symbols}{}

\begin{document}

\begin{frame}
\vspace{25pt}
\begin{center}
\LARGE{
{\color{DarkGreen}{
  A Pointfree Yoneda Lemma\\
  for\\
  Endofunctors\\
  of\\
  Functional Categories}\
}
}
\end{center}
\vspace{10pt}
\begin{center}
\LARGE{
{\color{DarkRed}{
  ICMSP 2023 \\ Madrid
}\
}
}
\end{center}
\vspace{10pt}
\begin{center}
\LARGE{
{\color{DarkBlue}{
Luc Duponcheel}}}
\end{center}
\end{frame}

\begin{frame}[fragile]
\frametitle{\begin{center}Previous life\end{center}}
\begin{itemize}
\item<2-> Mathematician
\item<3-> Programmer 
\vspace{20pt}
\begin{itemize}
\item<4-> Monad and monad transformer contributor 30 years ago
with Erik Meijer, Graham Hutton and Doaitse Swierstra at University of Utrecht
\end{itemize}
\end{itemize}
\end{frame}

\begin{frame}[fragile]
\frametitle{\begin{center}Current life\end{center}}
\begin{itemize}
\item<2-> Cyclist
\item<3-> Gardener 
\vspace{20pt}
\item<4-> Mathematician
\item<5-> Programmer
\end{itemize}
\end{frame}

\begin{frame}[fragile]
\frametitle{\begin{center}Main themes\end{center}}
\begin{itemize}
\item<2-> Mathematics $\rightarrow$ Programming
\vspace{20pt}
\item<3-> Programming $\rightarrow$ Mathematics
\end{itemize}
\end{frame}

\begin{frame}[fragile]
\frametitle{\begin{center}Mathematics $\rightarrow$ Programming\end{center}}
\begin{itemize}
\item<2-> Category theory
\begin{itemize}
\item<3-> Composition
\item<4-> Additional features 
\begin{itemize}
\item<5-> Transformation \ldots   
\end{itemize}
\vspace{20pt}
\item<6-> Separation
\begin{itemize}
\item<7-> specifications
\item<8-> implementations
\end{itemize}
\end{itemize}
\end{itemize}
\end{frame}

\begin{frame}[fragile]
\frametitle{\begin{center}Main sub-themes\end{center}}
\begin{itemize}
\item<2-> Composition : components
\begin{itemize}
\item<3-> Closed (pointfree) components
\item<4-> Open (pointful) components
\end{itemize}  
\vspace{20pt}
\item<5-> Additional features and separation : side-effects
\begin{itemize}
\item<6-> Specified (declared) side-effects 
\item<7-> Implemented (defined) side-effects 
\end{itemize}
\end{itemize}  
\end{frame}

\begin{frame}[fragile]
\frametitle{\begin{center}Pointful Effectfree Defined\end{center}}
\begin{itemize}
\item<2-> Expressions (pointful effectfree components)
\item<3-> Operational (are evaluated to yield a result)
\begin{itemize}
\vspace{20pt}
\item<4->
\begin{verbatim}
val zero = 0  
val one = zero + 1
val two = one + 1
two
\end{verbatim}  
\end{itemize}
\end{itemize}
\end{frame}

\begin{frame}[fragile]
\frametitle{\begin{center}Pointfree Effectfree Defined\end{center}}
\begin{itemize}
\item<2-> Functions (pointfree effectfree components)
\item<3-> Denotational (are meaningful)
\begin{itemize}
\vspace{20pt}   
\item<4->
\begin{verbatim}
def incrementF: Function[Int, Int] = 
  i => i + 1

def zeroF: Function[Unit, Int] =
  i => 0
\end{verbatim}
\vspace{20pt}   
\item<5->
\begin{verbatim}
incrementF o incrementF o zeroF
\end{verbatim}  
\end{itemize}
\end{itemize}
\end{frame}

\begin{frame}[fragile]
\frametitle{\begin{center}Pointful Effectful Defined\end{center}}
\begin{itemize}
\item<2->
\begin{verbatim}
val readInt: Int = ...
\end{verbatim}
\begin{itemize}
\vspace{20pt}
\item<3->
\begin{verbatim}
val zero = readInt  
val one = zero + 1
val two = one + 1
two
\end{verbatim}  
\end{itemize}
\end{itemize}
\end{frame}

\begin{frame}[fragile]
\frametitle{\begin{center}Pointfree Effectful Defined\end{center}}
\begin{itemize}
\item<2->
\begin{verbatim}
val readIntF: Function[Unit, Int] = ...
\end{verbatim}
\begin{itemize}
\vspace{20pt}
\item<3->
\begin{verbatim}
incrementF o incrementF o readIntF
\end{verbatim}  
\end{itemize}
\end{itemize}
\end{frame}

\begin{frame}[fragile]
\frametitle{\begin{center}Pointful Effectful Declared\end{center}}
\begin{itemize}
\item<2-> Computations (generalize expressions)  
\item<3-> Operational (are executed to yield a result)
\vspace{20pt}
\item<4->
\begin{verbatim}
val readIntC: C[Int]
\end{verbatim}
\begin{itemize}
\vspace{20pt}
\item<5->
\begin{verbatim}
readIntC bind { i =>
  result(i + 1) bind { j =>
    result(j + 1) bind { k => 
      result(k) 
    } 
  }
}
\end{verbatim}  
\end{itemize}
\end{itemize}
\end{frame}

\begin{frame}[fragile]
\frametitle{\begin{center}Pointfree Effectful Declared\end{center}}
\begin{itemize}
\item<2-> Programs (generalize functions)  
\begin{itemize}
\item<3-> Functions can be used as programs  
\end{itemize}
\item<4-> Denotational (are meaningful)
\vspace{20pt}
\item<5->
\begin{verbatim}
val readIntP: Program[Unit, Int]

val incrementP: Int --> Int = incrementF asProgram  
\end{verbatim}
\begin{itemize}
\vspace{20pt}
\item<6->
\begin{verbatim}
incrementP o incrementP o readIntP
\end{verbatim}  
\end{itemize}
\end{itemize}
\end{frame}

\begin{frame}[fragile]
\frametitle{\begin{center}Programming $\rightarrow$ Mathematics\end{center}}
\begin{itemize}
\item<2-> Type theory (as implemented by type systems)
\vspace{20pt}
\begin{itemize}
\item<3-> Proofs syntactic correctness
\item<4-> Provides confidence in semantic correctness
\vspace{20pt}
\begin{itemize}
\item<5-> Formulations of laws for specifications and proofs of for implementations 
\item<6-> Formulations and proofs of lemmas, propositions, theorems \ldots for specifications
\end{itemize}
\end{itemize}
\end{itemize}
\end{frame}

\begin{frame}[fragile]
\frametitle{\begin{center}Advancing insights\end{center}}
\begin{itemize}
\item<2-> Common patterns leading to auxiliary lemmas, propositions, theorems \ldots
\begin{itemize}
\item<3-> Genericity (Do not repeat yourself)
\end{itemize}
\item<4-> Recognition of known concepts
\begin{itemize}
\item<5-> Reusability (Embrace what exists)
\end{itemize}
\item<6-> My experience
\begin{itemize}
\item<7-> 25 percent Imagination (idea)
\item<8-> 25 percent Information (knowledge)
\item<9-> 50 percent Transpiration (the heavy lifting)
\end{itemize}
\vspace{20pt}
\item<10-> Never be happy with a solution, try to go for the best one
\end{itemize}
\end{frame}



\end{document}